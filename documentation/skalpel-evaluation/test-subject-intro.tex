% Copyright 2013 Heriot-Watt University
%
% Permission is granted to copy, distribute and/or modify this document
% under the terms of the GNU Free Documentation License, Version 1.3 or
% any later version published by the Free Software Foundation; with no
% Invariant Sections, no Front-Cover Texts, and no Back-Cover Texts.  A
% copy of the license is included in the section entitled "GNU Free
% Documentation License".

\documentclass{article}

\setlength{\parskip}{10pt}

\author{John Pirie}
\title{Participating in an Experiment to Evaluate the Effectiveness of Skalpel}
\date{\today}

\begin{document}

\maketitle

Thanks for agreeing to take part in an experiment to evaluate the
effectiveness of Skalpel, a type error diagnosis tool! This sheet
provides some information you need to know.

This experiment consists of 14 small test cases, which you will work
through one by one. The first four are warm-up tests, which your test
invigilator will work through with you, and allow you to familiarize
with the experiment framework. The experiment tests are split into two
groups, with half of them you will solve any type errors you make with
the aid of the compiler, and the other half with the aid of
Skalpel. All syntax errors are reported by Skalpel. With the exception
of the warm up tests, no hints can be provided.

You may write any additional helper functions you wish, but no basis
library functions may be used unless it is stated at the top of the
test file that this is permitted, e.g. List.map. No other X server
windows should be opened during the test, and interaction should be
made with the terminal running the experiment framework and the basic
text editor only. Most test files have regions of code which you may
not edit, and these are clearly marked. Please do not edit code in
these regions, or do something essentially equivalent (such as
redefine the function with slightly different semantics later in the
code).

You may skip a test after a period of time attempting the problem has
passed, simply ask the test invigilator. You may also stop the
experiment at any time, and you do not need to supply a reason.

A small SML cheat sheet is given below.

\begin{tabular}{ll}
Empty list & []\\
List with one element (x) & [x]\\
List construction & ::\\
String concatenation operator & \^{} \\
Append list y to list x & x@y\\
Pattern for decomposing a list into a head and a tail & (h::t)\\
Tuple notation: & $(x_1, x_2, ..., x_n)$\\
String & \texttt{"This is a string"} \\
Type variable & 'a\\
Identity function & (fn x =$>$ x)\\
\end{tabular}

\noindent Function: String.implode\\
Signature: char list -$>$ string\\
Description: Generates a string from the argument.\\

\noindent Function: String.explode\\
Signature: string -$>$ char list\\
Description: Returns a list of characters in the argument string.\\

\noindent Function: List.rev\\
Signature: 'a list -$>$ 'a list\\
Description: Reverses a list.\\

\noindent Function: Int.toString\\
Signature: int -$>$ string\\
Description: Returns the integer argument as a string.\\
\end{document}